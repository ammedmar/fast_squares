
\section{Introduction}

For effective computations involving topological spaces, discrete models are indispensable.
The category of simplicial complexes provides models not only for spaces but also, through the simplicial approximation theorem, for continuous maps between them.
We can obtain from these combinatorial models algebraic ones using the simplicial chains construction $\chains$.
In this algebraic representation, invariants of spaces such as their betti numbers ---the ranks of its cohomology with field coefficients--- can be readily computed through linear algebra alone.

In this article we focused on finer invariants of spaces enriching their mod 2 cohomology and going beyond betti numbers.
We are referring to the celebrated Steenrod squares
\begin{equation*}
Sq^k \colon H^\bullet(X; \F_2) \to H^\bullet(X; \F_2).
\end{equation*}
These can be thought of as arising from the broken $\S_2$-symmetry of the diagonal map
\begin{equation*}
\begin{tikzcd}[column sep=small, row sep=-3pt]
X \arrow[r] & X \times X \\
x \arrow[r, mapsto] & (x,x)
\end{tikzcd}
\end{equation*}
occurring during the passage from continuous objects to discrete/algebraic models.

For simplicial complexes, effective constructions of Steenrod squares have been known since their introduction in Steenrod's seminal 1947 paper \cite{steenrod47}.
They all rely on the construction of \textit{cup-$i$ coproducts}
\begin{equation*}
\Delta_i \colon \chains(X; \F_2)  \to \chains(X; \F_2)^{\otimes 2}\,,
\end{equation*}
i.e., natural linear maps satisfying
\begin{equation} \label{e:boundary of cup-i}
(1 + T) \Delta_{i-1} = \partial \circ \Delta_i + \Delta_i \circ \partial
\end{equation}
for every integer $i$, and that $\Delta_0$ is a chain approximation to the diagonal of $X$.

We will also consider \textit{cup-$i$ products}
\begin{equation*}
\smallsmile_i \colon \cochains(X; \F_2)^{\otimes 2} \to \cochains(X; \F_2),
\end{equation*}
the maps on cochains induced by linear duality.

Several constructions of cup-$i$ coproducts have been given in the literature starting with Steenrod's original formulas.
These include those resulting from the approach of Real \cite{real1996computability} and Gonz\'alez-D\'iaz--Real \cite{gonzalez1999combinatorial, gonzalez2003computation, gonzalez2005hpt} based on the EZ-AW chain contraction, the operadic methods of McClure-Smith \cite{mcclure03cochain} and Berger-Fresse \cite{berger04combinatorial}, and the prop viewpoint of the author \cite{medina2020prop1, medina2018prop2}.
For different tasks, some of these definitions present advantages although it has been shown \cite{medina2018axiomatic} that they all define isomorphic cup-$i$ coproducts.
An important task to consider, is the effective computation of Steenrod squares on simplicial complexes.
For this, and algorithm based on \cite{gonzalez1999combinatorial} was implemented in the open-source mathematics system \verb|SAGE| by John Palmieri \cite{sagemath}.

In this article we introduce a new description of Steenrod's cup-$i$ coproducts.
We are particularly interested in simplicial complexes, but we remark that our formulas apply to the more general context of simplicial sets, a categorical closure of simplicial complexes that includes, for example, the singular cochains of a topological space.

For simplicial complexes we use our formulas to introduce a new algorithm computing Steenrod square.
Additionally, we report a performance comparison between a \verb|Python| implementation of our algorithm and the one in \verb|SAGE|.

The speed gained with our algorithm is essential for the incorporation of Steenrod squares into persistence homology \cite{medina2018persistence}, a technique typically used in highly intensive data analysis tasks, for example \cite{carlsson2008images, carlsson2013viral, lee2018nanoporous}, and for which various softwares have been developed, including \cite{bauer2019ripser, gudhi, medina2020giottotda}.

On the theoretical side, our formulas can serve as a template for the definition of mod 2 cohomology operations in other contexts.
For example, Cantero-Mor\'an \cite{cantero2020khovanov} followed this approach to define Steenrod squares in Khovanov homology.

Other areas besides topology where cup-$i$ coproducts and their dual products have become important include: condensed matter physics, where they are used to describe action functionals of discrete topological field theories \cite{gaiotto2016spin, bhardwaj2017state, kapustin2017fermionic}, higher category theory, where they can be used to deduce the nerve of $\omega$-categories \cite{medina2020globular}, and convex geometry, where they are connected to projections of cyclic polytopes \cite{kapranov1991combinatorial}.
It is our hope that the new description of cup-$i$ products presented in this article can serve to facilitate new connections and to deepen those already existing.

In \cref{s:preliminaries} we review the notions from equivariant homological algebra and simplicial topology needed to present, in \cref{s:squares}, a definition of Steenrod squares in terms of cup-$i$ products.
We introduce our new formulas in \cref{s:formulas}, and our algorithm for the computation of Steenrod squares in \cref{s:algorithm}; providing a proof-of-concept comparison using \verb|SAGE| in \cref{s:comparison}.
In \cref{s:proof} we prove that our new formulas define cup-$i$ coproducts, and in \cref{s:outlook} we discuss finer invariants associated to Steenrod squares.