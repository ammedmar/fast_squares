
\section{Introduction}

For effective computations involving topological spaces, discrete models are indispensable.
The category of simplicial complexes models not only spaces but, through the simplicial approximation theorem, also continuous maps between them.
This combinatorial model leads to an algebraic one using the functor of simplicial chains, on which invariants such as the cohomology of spaces can be readily computed through linear algebra.

In this note we focused on further structure on cohomology of spaces arising from the broken symmetry of any chain approximation to the diagonal map of spaces $X \to X \times X$, the so called Steenrod squares $Sq^k \colon H^\bullet(X; \F_2) \to H^\bullet(X; \F_2)$.
The deeper reason for the existence of these invariants is the close connection between the objects representing mod-2 cohomology and the group $\S_2$, which acts by transposition on the codomain of the diagonal and leaves its image invariant.

Effective constructions of Steenrod squares have been known since the introduction of this cohomology operations in 1947 \cite{steenrod47}.
They all rely on the construction of so called cup-$i$ coproducts.
These are natural linear maps
\begin{equation*}
\Delta_i \colon \chains(X; \F_2)  \to \chains(X; \F_2)^{\otimes 2}
\end{equation*}
satisfying $\Delta_i = 0$ for $i < 0$ and
\begin{equation*}
(1 + T) \Delta_{i-1} = \partial \circ \Delta_i + \Delta_i \circ \partial.
\end{equation*}
The cup-$i$ coproducts can be interpreted as defining a coherent homotopical correction to the broken symmetry of the diagonal chain approximation $\Delta_0$.

The formulas for these coproducts introduced by Steenrod manifestly appeared in the more general context of $E_\infty$-operads and props in the work of McClure-Smith \cite{mcclure03cochain}, Berger-Fresse \cite{berger04combinatorial}, and the author \cite{medina2020prop1,medina2018prop2}.
A different construction of cup-$i$ coproducts was given earlier by Real and Gonz\'alez-D\'iaz--Real, based on the EZ-AW chain contraction and is implemented in SAGE.

In this article we introduce a new description of cup-$i$ coproducts which is in a sense dual to that given by Steenrod.
It allows us to present an algorithm for the computation of a cocycle representative of $\Sq^k([\alpha])$ that is quadratic on the number of basis elements defining the cocycle $\alpha$.

The fact these different constructions of cup-$i$ products give rise to the same cohomology operations follows from $\Delta_0$ being a chain approximation to the diagonal.
It is surprising though that they are all isomorphic at the chain level \cite{medina2018axiomatic}, which points to the combinatorial universality of Steenrod cup-$i$ products.
See \cite{medina2020globular} for a derivation of the $n$-categorical nerve from it and, through this result, a connection with polyhedral geometry as explained in \cite{bibid}.

Our new description of the cup-$i$ coproducts has bearing not only on applied topology.
It can also serve as a template for the definition of mod-2 cohomology operations on other homology theories, see for example \cite{bibid} where they are the basis for the definition of mod-2 operations on Khovanov homology.