
\section{New formulas} \label{s:formulas}

Let $X$ be a simplicial complex and $x \in X_n$. For a set
\begin{equation*}
U = \{u_1 < \dots < u_r\} \subseteq \{0, \dots, n\}
\end{equation*}
we use the notation $d_U(x) = d_{u_1} \ldots\, d_{u_r}(x)$.

\begin{definition} \label{d:cup-i coproducts}	
	For a non-negative integer $i$, the \textit{cup-$i$ coproduct}
	\begin{equation*}
	\Delta_i : C_\bullet(X; \F) \to C(X; \F)^{\otimes 2}_\bullet
	\end{equation*}
	is the linear map defined on a basis element $x \in X_n$ by
	\begin{equation} \label{equation: simplicial cup-i coproducts}
	\Delta_i(x) = \sum_U d_{U^+}(x) \otimes d_{U^-}(x),
	\end{equation}
	where the sum is taken over all sets $U = \{u_1 < \cdots < u_{n-i}\}$ with $u_j \in \{0, \dots, n\}$ and
	\begin{equation*}
	U^+ = \{u_j \in U\ |\ u_j \equiv j \text{ mod } 2\}, \qquad
	U^- = \{u_j \in U\ |\ u_j \not\equiv j \text{ mod } 2\}.
	\end{equation*}
	For a negative integer $i$ we set $\Delta_i = 0$.
\end{definition}

\begin{lemma}
	This construction is natural with respect to simplicial maps.
	Explicitly, if $f$ is a simplicial map, then, for every $i \in \Z$,
	\begin{equation*}
	\Delta_i \circ f_\bullet = (f_\bullet \otimes f_\bullet) \circ \Delta_i.
	\end{equation*}
\end{lemma}

\begin{proof}
	content...
\end{proof}

\begin{theorem} \label{t:main}
	For any integer $i$,
	\begin{equation} \label{eq: cup-i coproducts boundary relation}
	\partial \circ \Delta_{i} + \Delta_{i+1} \circ \partial = (1 +T ) \Delta_{i-1}.
	\end{equation}
\end{theorem}

We devote Section~\ref{s:proof} to the proof of this theorem.

\begin{example}
	AW diagonal and $Sq^0$.
	
	...
\end{example}