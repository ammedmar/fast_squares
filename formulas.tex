
\section{New formulas and algorithm} \label{s:new stuff}

Let $X$ be a simplicial complex and $x \in X_n$. For a set
\begin{equation*}
U = \{u_1 < \dots < u_r\} \subseteq \{0, \dots, n\}
\end{equation*}
we use the notation $d_U(x) = d_{u_1} \ldots\, d_{u_r}(x)$.

\begin{definition} \label{d:cup-i coproducts}	
	For a non-negative integer $i$, the \textit{cup-$i$ coproduct}
	\begin{equation*}
	\Delta_i : C_\bullet(X; \F) \to C(X; \F)^{\otimes 2}_\bullet
	\end{equation*}
	is the linear map defined on a basis element $x \in X_n$ by
	\begin{equation} \label{equation: simplicial cup-i coproducts}
	\Delta_i(x) = \sum_U d_{U^+}(x) \tensor d_{U^-}(x),
	\end{equation}
	where the sum is taken over all sets $U = \{u_1, \dots, u_{n-i}\}$ with $u_j \in \{0, \dots, n\}$ and
	\begin{equation*}
	U^+ = \{u_j\ |\ u_j \equiv j \text{ mod } 2\}, \qquad
	U^- = \{u_j\ |\ u_j \not\equiv j \text{ mod } 2\}.
	\end{equation*}
\end{definition}

\begin{theorem} \label{t:main theorem}
	For any integer $i$,
	\begin{equation} \label{eq: cup-i coproducts boundary relation}
	\partial \circ \Delta_{i} + \Delta_{i+1} \circ \partial = (1 +T ) \Delta_{i-1}.
	\end{equation}
\end{theorem}

We devote the next section to the proof of this theorem.
Assuming it for the moment, we now describe an algorithm for the determination of a cocycle representative of $Sq^k([\alpha])$ given a representative cocycle $\alpha$.

Let us represent the cocycle $\alpha$ as a set of $n$-simplices $A = \{a_1, \dots, a_m\} = \{a \in X_n \mid \alpha(a) = 1\}$.


... as presented in Figure~\ref{f:algorithm}.

\begin{figure}
	\begin{algorithm} [H]
	\KwIn{$A = \{a_1, \dots, a_m\} \subseteq X_d$}
	$B = \emptyset$\\
	\ForAll{$a_i\ \mathrm{and}\ a_j\ \mathrm{with}\ i < j$}
	{	
		$a_{ij} = a_i \cup a_j$\\
		\If{$a_{ij} \in X_{d+k}$}
		{
			$\overline{a}_i = a_i \setminus a_j$\ ;\ \ 
			$\overline{a}_j = a_j \setminus a_i$\ ;\ \
			$\overline{a}_{ij} = \overline{a}_i \cup \overline{a}_j$ \\
			$index =$ empty dictionary\\
			\ForAll {$v \in \overline{a}_{ij}$}
			{
				$p = \mathrm{position\ of\ } v \mathrm{\ in\ } a_{ij}$\ ;\ \
				$\overline{p} = \mathrm{position\ of\ } v \mathrm{\ in\ } \overline{a}_{ij}$\\
				$index[v] = {p}\, +\, \overline{p}\ \ \mathrm{residue\ mod}\ 2$
			}
			\If{\hspace*{3pt}\{$index[v] \mid \mathrm{in}$ $\overline{a}_i$\} $\mathrm{xor}$ \{$index[v] \mid v$ $\mathrm{in}$ $\overline{a}_j$\} = $\{0,1\}$}
				{$B = B$ xor $\{a_{ij}\}$}
		}		
	}
	\KwOut{$B$}
	\caption{$SQ^k$}
\end{algorithm}
	\caption{TBW}
	\label{f:algorithm}
\end{figure}