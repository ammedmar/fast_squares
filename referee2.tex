\documentclass{amsart}
\usepackage{amsmath}
\input{aux/style}
\usepackage{graphicx}
\usepackage{framed}
\usepackage[backend=biber,
style=alphabetic,
backref=true,
url=false,
doi=false,
isbn=false,
eprint=false
]{biblatex}

\addbibresource{aux/usualpapers.bib}
\addbibresource{aux/mypapers.bib}

\begin{document}
	\thispagestyle{empty}
	\begin{center}
		\Huge{Second replay to reviewer 2}
		\large{}
	\end{center}

	\vspace*{5pt}
	\noindent\rule{\textwidth}{1pt}

	\noindent\textbf{Reviewer 2}:
	I am overall satisfied with the answers and corrections made by the authors.

	I only need one clarification in the answer of my first question.
	The authors claim that ``We also mention that the isomorphism between Steenrod's and Gonzales-Diaz--Real's constructions was, to the best of our knowledge, missing from the literature prior to our forthcoming axiomatic work."

	What I think is that Medina-Mardones and, for example, Gonzalez-Diaz-Real's formulas are exactly the same (not only isomorphic).
	For example, if you compute $\Delta_1$ using Medina-Mardones's formula, don't you get:
	\[
	d_{i0+1}...d_{i1-1} \otimes d0...d_{i0-1}d_{i1+1}...dm ?
	\]

	Maybe I am wrong, and this is why I suggest to the author to add the explicit expression for some cases, for example $\Delta_1$ and $\Delta_2$ and see the differences between Medina-Mardones and Gonzalez-Diaz-Real's formulas.
	Otherwise, the claim that you ''present new formulas defining a cup-i construction" is not totally convincing.

	\noindent\rule{\textwidth}{1pt}

	\newpage
	\noindent\textbf{Author}:
	I stand by the claim that this paper ``presents new formulas defining a cup-$i$ construction." I have segmented the explanation in three parts.

	\subsection*{1} Consider Gonzalez-Diaz--Real's formulas:
	\begin{center}
		\vspace*{-10pt}
		\boxed{\includegraphics[width=\textwidth]{real.png}}
	\end{center}
	Consider my formulas in the same notation as above:
	\begin{framed}
		\noindent$\bullet$ If $i \leq j$, then
		\[
		Sq^i(c)(x) = (c \otimes c) \sum \partial_{U^0}(x) \otimes \partial_{U^1}(x)
		\]
		where the sum is taken over all $U = \{u_1 < \cdots < u_{j}\} \subseteq \{0, \dots, i+j\}$ and
		\begin{equation*}
			U^0 = \{u_k \in U \mid u_k \equiv k \text{ mod } 2\}, \qquad
			U^1 = \{u_k \in U \mid u_k \not\equiv k \text{ mod } 2\}.
		\end{equation*}
		\noindent$\bullet$ If $i > j$, then $Sq^i(c)(x) = 0$.
	\end{framed}
	Gonzalez-Diaz--Real's formulas need to treat two distinct cases based on the parity of $i+j$ and are essentially recursive.
	Mine are presented in closed form and without cases.

	\subsection*{2} Let me expand on the distinction between an object defined by certain formulas and the formulas themselves.
	Given $n \in \mathbb N$, consider the following definitions of an integer: $a$) the number of triangulations of the $n$-gon, or $b$) the number of different ways $n+1$ factors can be completely parenthesized.
	These are the same integer but the formulas defining it are not.

	It turns out that, as the referee verified in low dimensions, the formulas above (Gonzalez-Diaz--Real's and mine) define the \textit{same} cup-$i$ products.
	Proving this is difficult and requires the full power of my axiomatic characterization \cite{medina2022axiomatic}.

	\subsection*{3} Finally, if the reviewer does not consider my formulas to be different from Gonzalez-Diaz--Real's, then he or she cannot consider Gonzalez-Diaz--Real's formulas different from Steenrod's.
	This is because, as I proved using my axiomatic characterization and can be easily verified in low dimensions, they define the \textit{same} cup-$i$ products.

	\sloppy
	\printbibliography

\end{document}