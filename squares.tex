
\section{Steenrod squares} \label{s:squares}

Let $\F_2$ be the field with two elements and $\S_2$ the group with only one non-identity element $T$.
In this section we define for any simplicial complexes $X$ and every integer $k$ the Steenrod $k^{\mathrm{th}}$ square
\begin{equation*}
Sq^k \colon H^\bullet(X; \F_2) \to H^{\bullet}(X; \F_2).
\end{equation*}
Consider the chain complex
\begin{equation*}
\begin{tikzcd}[column sep=normal]
& W =  \F_2[\S_2]\{e_0\} & \arrow[l, "1+T"'] \F_2[\S_2]\{e_1\} & \arrow[l, "1+T"']
\F_2[\S_2]\{e_2\} & \arrow[l, "1+T"'] \cdots
\end{tikzcd}
\end{equation*}
with its natural $\S_2$-action.
A collection $\{\Delta_i\}_{i \in \Z}$ of linear maps from $\chains \to \chains^{\otimes 2}$ satisfying $\Delta_i = 0$ for $i < 0$ and
\begin{equation*}
(1 + T) \Delta_{i-1} = \partial \circ \Delta_i + \Delta_i \circ \partial
\end{equation*}
is, using the adjuntion isomorphism, the same as an $\S_2$-equivariant chain map
\begin{equation} \label{e:steenrod diagonal}
W \otimes C_\bullet \to C_\bullet^{\otimes 2}.
\end{equation}
We refer to any non-zero such map that is natural with respect to simplicial maps, i.e., making the diagram
\begin{equation*}
\begin{tikzcd}
W \otimes \chains(X) \arrow[r] \arrow[d, "\id \otimes f_\bullet"] & \chains(X)^{\otimes 2} \arrow[d, "f_\bullet \otimes f_\bullet"] \\
W \otimes \chains(X) \arrow[r]& \chains(X)^{\otimes 2}
\end{tikzcd}
\end{equation*}
commute for any simplicial map $f \colon X \to Y$, as a \textit{cup-$i$ construction} and to the map $\Delta_i$ as its associated \textit{cup-$i$ coproduct}.

Using the linear duality functor on the map \eqref{e:steenrod diagonal} assumed to be a cup-$i$ construction and passing to fix points gives a chain map
\begin{equation*}
\begin{tikzcd}
Hom\left(C_\bullet \otimes C_\bullet, \F_2 \right)^{\mathrm S_2} \arrow[r] &
Hom\left(W(2) \otimes C_\bullet, \F_2 \right)^{\mathrm S_2},
\end{tikzcd}
\end{equation*}
which we can complete, using the adjunction isomorphisms of Section \ref{s:preliminaries}, to a commutative diagram
\begin{equation*}
\begin{tikzcd}
Hom\left(C_\bullet \otimes C_\bullet, \F_2 \right)^{\mathrm S_2} \arrow[r] &
Hom\left(W(2) \otimes C_\bullet, \F_2 \right)^{\mathrm S_2} \arrow[d] \\
\left(C^\bullet \otimes C^\bullet\right)^{\mathrm S_2} \arrow[u]&
Hom\left(W(2)_{\mathrm S_2} \otimes C_\bullet, \F_2 \right) \arrow[d] \\
C^\bullet \arrow[u, "doubleing"] \arrow[r, dashed]&
Hom\left(W(2)_{\mathrm S_2}, C^\bullet\right),
\end{tikzcd}
\end{equation*}
where the choice of coefficients ensures the \textit{doubleing map} $\alpha \mapsto \alpha \otimes \alpha$ is linear.
By adjunction, the dashed arrow defines a linear map
\begin{equation} \label{e:Steenrod squares parameterized}
\begin{tikzcd}[row sep=tiny, column sep = tiny]
C^\bullet \otimes W(2)_{\mathrm S_2} \arrow[r] &[-10pt] C^\bullet \\
\alpha \otimes e_i \arrow[r, |->] & (\alpha \otimes \alpha)\Delta_i(-)
\end{tikzcd}
\end{equation}
descending to mod $2$ cohomology.
The \textit{Steenrod squares}, whose importance in homotopy theory is hard to overstate, are defined by reindexing this map.
Explicitly,
\begin{equation*}
\begin{tikzcd}[row sep=tiny, column sep=tiny]
Sq^k \colon H^{-n} \arrow[r] & H^{-n-k} \\
\phantom{Sq^k \colon}{[\alpha]} \arrow[r, |->] & {\left[(\alpha \otimes \alpha)\Delta_{n-k}(-) \right]}.
\end{tikzcd}
\end{equation*}

Although we do not use the algebra structure on the mod 2 cohomology of spaces in this article, we remark that the linear dual of $\Delta_0$ represents it, explicitly, if $[\alpha], [\beta] \in H^\bullet$ then $[\alpha][\beta] = [\alpha \smallsmile_0 \beta]$, in particular, if $[\alpha]$ is of degree $-k$ then $\Sq^k\big([\alpha]\big) = [\alpha] [\alpha]$, which motivates the name of these operations.

For the reader familiar with group homology, we remark that Steenrod squares are parameterized by classes on the mod $2$ homology of $\mathrm S_2$.
Steenrod later used this viewpoint to define non-constructively for any prime $p$ operations on the mod $p$ cohomology of spaces \cite{steenrod1962cohomology}.
Analogues of the $\Delta_i$ maps for odd primes constructively defining these operations were given in \cite{medina2020odd}.