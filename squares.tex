
\section{Steenrod squares} \label{s:steenrod_squares}

Let $\F_2$ be the field with two elements and $\S_2$ the group with only one non-identity element $T$.
In this section we define for any simplicial complexes $X$ and every integer $k$ the Steenrod $k^{\mathrm{th}}$ square
\begin{equation*}
Sq^k \colon H^\bullet(X; \F_2) \to H^{\bullet}(X; \F_2).
\end{equation*}
Consider the chain complex
\begin{equation*}
\begin{tikzcd}[column sep=normal]
& W =  \F_2[\S_2]\{e_0\} & \arrow[l, "1+T"'] \F_2[\S_2]\{e_1\} & \arrow[l, "1+T"']
\F_2[\S_2]\{e_2\} & \arrow[l, "1+T"'] \cdots
\end{tikzcd}
\end{equation*}
with its natural $\S_2$-action, and notice that a collection $\{\Delta_i\}_{i\in \Z}$ of linear maps from $\chains \to \chains^{\otimes 2}$ satisfying $\Delta_i = 0$ for $i < 0$ and
\begin{equation*}
(1 + T) \Delta_{i-1} = \partial \circ \Delta_i + \Delta_i \circ \partial
\end{equation*}
is the same data as an $\S_2$-equivariant chain map
\begin{equation} \label{e: steenrod diagonal}
W \otimes C_\bullet \to C_\bullet^{\otimes 2}.
\end{equation}
Using the linear duality functor on the map \eqref{e: steenrod diagonal} and passing to fix points we have a chain map
\begin{equation*}
\begin{tikzcd}
Hom\left(C_\bullet \otimes C_\bullet, \F \right)^{\mathrm S_2} \arrow[r] &
Hom\left(W(2) \otimes C_\bullet, \F \right)^{\mathrm S_2},
\end{tikzcd}
\end{equation*}
which we can complete, using the adjuntion isomorphisms of Section \ref{sec: preliminaries}, to a commutative diagram
\begin{equation*}
\begin{tikzcd}
Hom\left(C_\bullet \otimes C_\bullet, \F \right)^{\mathrm S_2} \arrow[r] &
Hom\left(W(2) \otimes C_\bullet, \F \right)^{\mathrm S_2} \arrow[d] \\
\left(C^\bullet \otimes C^\bullet\right)^{\mathrm S_2} \arrow[u]&
Hom\left(W(2)_{\mathrm S_2} \otimes C_\bullet, \F \right) \arrow[d] \\
C^\bullet \arrow[u, "doubleing"] \arrow[r, dashed]&
Hom\left(W(2)_{\mathrm S_2}, C^\bullet\right),
\end{tikzcd}
\end{equation*}
where the choice of coefficients ensures the \textit{doubleing map} $\alpha \mapsto \alpha \otimes \alpha$ is linear.
By adjunction, the dashed arrow defines a linear map
\begin{equation*}
\begin{tikzcd}[row sep=tiny, column sep = tiny]
C^\bullet \otimes W(2)_{\mathrm S_2} \arrow[r] &[-10pt] C^\bullet \\
\alpha \otimes e_i \arrow[r, |->] & (\alpha \otimes \alpha)\Delta_i(-)
\end{tikzcd}
\end{equation*}
descending to mod-$2$ cohomology. The Steenrod squares are defined by reindexing this map. Explicitly,
\begin{equation*}
\begin{tikzcd}[row sep=tiny, column sep=tiny]
Sq^k \colon H^{-n} \arrow[r] & H^{-n-k} \\
\phantom{Sq^k \colon}{[\alpha]} \arrow[r, |->] & {\left[(\alpha \otimes \alpha)\Delta_{n-k}(-) \right]}.
\end{tikzcd}
\end{equation*}
The importance of Steenrod squares in homotopy theory is hard to overstate.

\begin{remark}
	For the reader familiar with group homology, we remark that Steenrod squares are parameterized by classes on the mod-$2$ homology of $\mathrm S_2$.
	Steenrod also defined operation on the mod-$p$ homology of spaces for odd primes \cite{}, this are parameterized by the mod-$p$ homology of $\mathrm S_p$, and the analogue of the $\Delta_i$ maps of Steenrod where defined in \cite{medina2020odd}.
\end{remark}