%!TEX root = ../fast_sq.tex

\section{New formulas for cup-\texorpdfstring{$i$}{i} products} \label{s:formulas}

\anibal{Improve intro}

In this section we introduce our new description of a cup-$i$ construction.

\begin{notation}
	Let $X$ be a simplicial complex and $x \in X_n$.
	For a set
	\begin{equation*}
	U = \{u_1 < \dots < u_r\} \subseteq \{0, \dots, n\}
	\end{equation*}
	we write $d_U(x) = d_{u_1} \ldots\, d_{u_r}(x)$, with $d_{\emptyset}(x) = x$.
\end{notation}

\begin{definition} \label{d:cup-i coproducts}
	For any simplicial complex $X$ and integer $i$, the \textit{cup-$i$ coproduct}
	\begin{equation*}
	\Delta_i : C_\bullet(X; \Ftwo) \to C(X; \Ftwo)^{\otimes 2}_\bullet
	\end{equation*}
	is the linear map defined on a simplex $x \in X_n$ to be $0$ if $i \not\in \{0, \dots, n\}$ and is otherwise given by
	\begin{equation} \label{e:new formulas}
	\Delta_i(x) = \sum d_{U^0}(x) \otimes d_{U^1}(x)
	\end{equation}
	where the sum is taken over all subsets $U = \{u_1 < \cdots < u_{n-i}\} \subseteq \{0, \dots, n\}$ and
	\begin{equation} \label{e:partition subsets}
	U^0 = \{u_j \in U\mid u_j \equiv j \text{ mod } 2\}, \qquad
	U^1 = \{u_j \in U\mid u_j \not\equiv j \text{ mod } 2\}.
	\end{equation}
\end{definition}

\begin{example} \label{ex:alexander-whitney diagonal}
	For any $x \in X_n$ and $i = 0$ our formulas give
	\begin{equation*}
	\Delta_0(x) = \sum_{j=0}^n d_{j+1} \cdots d_{n}(x) \otimes d_{0} \cdots d_{j-1}(x),
	\end{equation*}
	a map known as \textit{Alexander--Whitney diagonal} and widely used to define the algebra structure on cohomology (\cref{r:cup product}).
\end{example}

\begin{example} \label{ex:Sq0 is the identity}
	For any simplex $x \in X_n$ our formulas give
	\begin{equation*}
	\Delta_n(x) = x \otimes x,
	\end{equation*}
	implying the well known fact that $Sq^0$ is the identity.
\end{example}

\begin{theorem} \label{t:main}
	The maps introduced in \cref{d:cup-i coproducts} define a cup-$i$ construction.
\end{theorem}

Since by the previous example $\Delta_0\big([v]\big) = [v] \otimes [v] \neq 0$, we need to check that each $\Delta_i$ is natural and satisfies \eqref{e:boundary of cup-i}.
We state these claims as two lemmas.

\begin{lemma} \label{l:naturality}
	For any simplicial map $f$ and integer $i$, the $i^\th$ map of \cref{d:cup-i coproducts} satisfies
	\begin{equation*}
	\Delta_i \circ f_\bullet = (f_\bullet \otimes f_\bullet) \circ \Delta_i.
	\end{equation*}
\end{lemma}

\begin{proof}
	Consider a simplex $x = [v_0, \dots, v_n]$ and first assume that $f_\bullet(x)$ is not $0$.
	Then, for any proper subset $U \subsetneq \{0, \dots, n\}$ the image of $d_U(x)$ is not $0$ as well and we have
	\begin{align*}
	\Delta_i \circ f_\bullet(x) =\ &
	\Delta_i \big([f(v_0), \dots, f(v_n)]\big) \\ =\ &
	\sum d_{U^0} \big([f(v_0), \dots, f(v_n)]\big) \otimes d_{U^1} \big([f(v_0), \dots, f(v_n)]\big) \\ =\ &
	(f_\bullet \otimes f_\bullet) \sum d_{U^0} \big([v_0, \dots, v_n]\big) \otimes d_{U^1} \big([v_0, \dots, v_n]\big) \\ =\ &
	(f_\bullet \otimes f_\bullet) \circ \Delta_i(x).
	\end{align*}
	If $f_\bullet(x) = 0$ then there exists consecutive elements $v_j$ and $v_{j+1}$ with $f(v_j) = f(v_{j+1})$.
	To prove that $(f_\bullet \otimes f_\bullet) \circ \Delta(x) = 0$ it suffices to show that for any $U \in \P_{n-i}(n)$ either the simplex $d_{U^0}(x)$ or $f d_{U^1}(x)$ contains both $v_j$ and $v_{j+1}$.
	If $U$ does not contain both $j$ and $j+1$ this is immediate.
	If it does, we have that $j, j+1 \in U^0$ or $j, j+1 \in U^1$ since they are consecutive.
\end{proof}

\begin{lemma} \label{l:main}
	For any integer $i$, the $i^\th$ map of \cref{d:cup-i coproducts} satisfies
	\begin{equation*}
	\partial \circ \Delta_{i} + \Delta_{i+1} \circ \partial = (1+T) \Delta_{i-1}.
	\end{equation*}
\end{lemma}

We devote \cref{s:proof} to the proof of this lemma, which, together with \cref{l:naturality}, imply \cref{t:main}.