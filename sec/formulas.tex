%!TEX root = ../fast_sq.tex

\section{New formulas for cup-\texorpdfstring{$i$}{i} products} \label{s:formulas}

In this section we introduce formulas which we show to define a cup-$i$ construction in \cref{s:proof}.
To the best of our knowledge these are new expressions.
In forthcoming work \cite{medina2018axiomatic} we prove that the resulting cup-$i$ construction agrees up to isomorphism with Steenrod's original and all other cup-$i$ constructions in the literature.

\begin{notation}
	Let $X$ be a simplicial complex and $x \in X_n$.
	For a set
	\begin{equation*}
	U = \{u_1 < \dots < u_r\} \subseteq \{0, \dots, n\}
	\end{equation*}
	we write $d_U(x) = d_{u_1}\! \dotsm \, d_{u_r}(x)$, with $d_{\emptyset}(x) = x$.
\end{notation}

\begin{definition} \label{d:cup-i coproducts}
	For any simplicial complex $X$ and integer $i$
	\begin{equation*}
	\Delta_i \colon C_\bullet(X; \Ftwo) \to C_\bullet(X; \Ftwo) \ot C_\bullet(X; \Ftwo)
	\end{equation*}
	is the linear map defined on a simplex $x \in X_n$ to be $0$ if $i \not\in \{0, \dots, n\}$ and is otherwise given by
	\begin{equation} \label{e:new formulas}
	\Delta_i(x) = \sum d_{U^0}(x) \ot d_{U^1}(x)
	\end{equation}
	where the sum is taken over all subsets $U = \{u_1 < \cdots < u_{n-i}\} \subseteq \{0, \dots, n\}$ and
	\begin{equation} \label{e:partition subsets}
	U^0 = \{u_j \in U\mid u_j \equiv j \text{ mod } 2\}, \qquad
	U^1 = \{u_j \in U\mid u_j \not\equiv j \text{ mod } 2\}.
	\end{equation}
\end{definition}

\begin{example} \label{ex:alexander-whitney diagonal}
	For any $x \in X_n$ and $i = 0$ our formulas give
	\begin{equation*}
	\Delta_0(x) = \sum_{j=0}^n d_{j+1} \cdots d_{n}(x) \ot d_{0} \cdots d_{j-1}(x),
	\end{equation*}
	a map known as \textit{Alexander--Whitney diagonal} and widely used to define the algebra structure on cohomology (\cref{r:cup product}).
\end{example}

\begin{example} \label{ex:Sq0 is the identity}
	For any simplex $x \in X_n$ our formulas give
	\begin{equation*}
	\Delta_n(x) = x \ot x,
	\end{equation*}
	implying, after \cref{t:main} below, the well known fact that $Sq^0$ is the identity.
\end{example}

\begin{theorem} \label{t:main}
	The maps introduced in \cref{d:cup-i coproducts} define a cup-$i$ construction.
\end{theorem}

\begin{remark}
	Two \mbox{cup-$i$} constructions, say $\triangle$ and $\triangle^\prime$, are \textit{isomorphic} if there is an automorphism $\phi$ of $W$ making the following diagram commute:
	\[
	\begin{tikzcd} [column sep = 0, row sep=normal]
	W \ot C_\bullet \arrow[rr, "\phi \ot \id"] \arrow[rd, in=180, out=-90, "\triangle^{\phantom{\prime}}"', near start] & &
	W \ot C_\bullet \arrow[ld, in=0, out=-90, "\triangle^\prime", near start] \\
	& C_\bullet \ot C_\bullet & .
	\end{tikzcd}
	\]
	The cup-$i$ products of Steenrod seem to be combinatorially fundamental.
	In forthcoming work \cite{medina2018axiomatic} that depends on \cref{t:main} we show, through an axiomatic characterization, that all known cup-$i$ constructions on simplicial chains are isomorphic -- and not just homotopic -- to the one introduced here.
	These constructions are: Steenrod's original \cite{steenrod1947products}, the one obtained using the $\EZ$-$\AW$ contraction \cite{real1996computability, gonzalez-diaz1999steenrod}, those from combinatorial operads \cite{mcclure2003multivariable, berger2004combinatorial}, and the one defined by the $\Med$-bialgebra structure on standard simplices \cite{medina2020prop1, medina2021prop2}.
	Furthermore, this cup-$i$ construction defines naturally another fundamental construction: the nerve of higher categories \cite{street1987orientals, medina2020globular}.
\end{remark}


In order to prove \cref{t:main} we need to check that each $\Delta_i$ is natural and satisfies \eqref{e:boundary of cup-i} -- \cref{ex:Sq0 is the identity} implies the non-degeneracy condition.
We state these claims as two lemmas.

\begin{lemma} \label{l:naturality}
	For any simplicial map $f$ and integer $i$ we have
	\begin{equation*}
	\Delta_i \circ f_\bullet = (f_\bullet \ot f_\bullet) \circ \Delta_i.
	\end{equation*}
\end{lemma}

\begin{proof}
	Consider a simplex $x = [v_0, \dots, v_n]$ and let $i \in \{0, \dots, n\}$, otherwise the identity holds trivially.
	First assume that $f_\bullet(x)$ is not $0$.
	Then, for any proper subset $U \subsetneq \{0, \dots, n\}$ the image of $d_U(x)$ is not $0$ as well and we have
	\begin{align*}
	\Delta_i \circ f_\bullet(x) =\ &
	\Delta_i \big([f(v_0), \dots, f(v_n)]\big) \\ =\ &
	\sum d_{U^0} \big([f(v_0), \dots, f(v_n)]\big) \ot d_{U^1} \big([f(v_0), \dots, f(v_n)]\big) \\ =\ &
	(f_\bullet \ot f_\bullet) \sum d_{U^0} \big([v_0, \dots, v_n]\big) \ot d_{U^1} \big([v_0, \dots, v_n]\big) \\ =\ &
	(f_\bullet \ot f_\bullet) \circ \Delta_i(x).
	\end{align*}
	If $f_\bullet(x) = 0$ then there exists consecutive elements $v_j$ and $v_{j+1}$ with $f(v_j) = f(v_{j+1})$.
	To prove that $(f_\bullet \ot f_\bullet) \circ \Delta_i(x) = 0$ it suffices to show that for any $U \in \P_{n-i}(n)$ either the simplex $d_{U^0}(x)$ or $d_{U^1}(x)$ contains both $v_j$ and $v_{j+1}$.
	If $U$ does not contain both $j$ and $j+1$ this is immediate.
	If it does, we have that $j, j+1 \in U^0$ or $j, j+1 \in U^1$ since they are consecutive implying $v_j, v_{j+1} \in d_{U^1}(x)$ in the first case and $v_j, v_{j+1} \in d_{U^0}(x)$ in the second.
\end{proof}

\begin{lemma} \label{l:main}
	For any integer $i$ we have
	\begin{equation*}
	\partial \circ \Delta_{i} + \Delta_{i+1} \circ \partial = (1+T) \Delta_{i-1}.
	\end{equation*}
\end{lemma}

We devote \cref{s:proof} to the proof of this lemma.
We now turn to the development of a fast method for the computation of Steenrod squares on the cohomology of finite simplicial complexes leveraging formula \eqref{e:new formulas}.