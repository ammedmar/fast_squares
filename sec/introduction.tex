%!TEX root = ../fast_sq.tex

\section{Introduction}

For effective computations involving topological spaces, discrete models are indispensable.
The category of simplicial complexes provides models not only for spaces but also, through the simplicial approximation theorem, for continuous maps between them.
We can obtain algebraic models from these using simplicial chains and their dual cochains.
Using these algebraic representations, invariants of spaces such as Betti numbers can be readily computed using linear algebra alone.

In this article we focused on finer invariants of spaces enriching their mod 2 cohomology and going beyond Betti numbers.
We are referring to the celebrated Steenrod squares
\begin{equation*}
Sq^k \colon H^\bullet(X; \Ftwo) \to H^\bullet(X; \Ftwo).
\end{equation*}
These operations can be thought of as arising from the broken $\Sym_2$-symmetry of the diagonal map
\begin{equation*}
\begin{tikzcd}[column sep=small, row sep=-3pt]
X \arrow[r] & X \times X \\
x \arrow[r, mapsto] & (x,x)
\end{tikzcd}
\end{equation*}
occurring during the passage from continuous descriptions to discrete/algebraic models.

For simplicial complexes, effective constructions of Steenrod squares have been known since their introduction in Steenrod's seminal 1947 paper \cite{steenrod1947products}.
They all rely on a \textit{cup-$i$ construction}, a structure on chains given by a collection of natural linear maps
\begin{equation*}
\Delta_i \colon C_\bullet(X; \Ftwo)  \to C_\bullet(X; \Ftwo)^{\otimes 2}\,,
\end{equation*}
satisfying for every integer $i$ that
\begin{equation*}
(1+T) \Delta_{i-1} =
\partial \circ \Delta_i + \Delta_i \circ \partial
\end{equation*}
where $T$ denoted the transposition of tensor factors,
and such that $\Delta_0$ is a chain approximation to the diagonal of $X$.
These \textit{cup-$i$ coproducts} and their linear dual products are important in their own right.
For example, in condensed matter physics they are used to describe action functionals of topological field theories \cite{gaiotto2016spin, bhardwaj2017state, kapustin2017fermionic}, in higher category theory, they can be used to deduce the nerve of $n$-categories \cite{medina2020globular}, and in convex geometry, they are connected to projections of cyclic polytopes \cite{kapranov1991combinatorial}.

In this article we introduce a new cup-$i$ construction. \anibal{misleading}
We are particularly interested in simplicial complexes, but our formulas apply in the more general context of simplicial sets, a categorical closure of simplicial complexes used, for example, to define the singular homology of topological spaces.

Several cup-$i$ constructions have been given in the literature starting with Steenrod's original formulas~\cite{steenrod1947products}.
These include those resulting from the approach of Real \cite{real1996computability} and Gonz\'alez-D\'iaz--Real \cite{gonzalez-diaz1999steenrod, gonzalez2003computation, gonzalez-diaz2005cocyclic} based on the EZ-AW chain contraction, the operadic methods of McClure-Smith \cite{mcclure2003multivariable} and Berger-Fresse \cite{berger2004combinatorial}, and the prop viewpoint of the author \cite{medina2020prop1, medina2018prop2}.
The question of comparing these different cup-$i$ constructions will be addressed via an axiomatic characterization in \cite{medina2018axiomatic}, where it is shown that all of these cup-$i$ constructions, including the one given here, are isomorphic and not just homotopic.

We highlight three uses of our formulas.
1) They are key to prove the axiomatic characterization of cup-$i$ products mentioned above.
2) In \cite{cantero2020khovanov}, Cantero-Mor\'an defined Steenrod squares in mod 2 Khovanov homology \cite{khovanov2000categorification} by reinterpreting them in the context of augmented semi-simplicial objects in the Burnside category.
3) They lead to fast computations of Steenrod square as we describe next.

For the effective computation of Steenrod squares on simplicial complexes, an algorithm based on \cite{gonzalez-diaz1999steenrod} was implemented in the open-source mathematics system \verb|SAGE| by John Palmieri \cite{sagemath}.
We use our formulas to introduce a new algorithm for this task, and present a performance comparison between a \verb|Python| implementation of our algorithm and the one in \verb|SAGE|.
The speed gained with our algorithm is essential for the incorporation of Steenrod squares into persistence homology \cite{medina2018persistence}, a technique typically used in highly intensive data analysis tasks \cite{carlsson2008images, carlsson2013viral, lee2018nanoporous} and for which various softwares exist \cite{bauer2019ripser, gudhi, medina2021giotto}.

\subsection*{Outline}

In \cref{s:preliminaries} we review the notions from equivariant homological algebra and simplicial topology needed to present, in \cref{s:squares}, the definition of cup-$i$ construction and Steenrod squares.
We introduce our new formulas in \cref{s:formulas}, and our algorithm in \cref{s:algorithm}; providing a proof-of-concept comparison using \verb|SAGE| in \cref{s:comparison}.
In \cref{s:proof}, the technical core of this article, we prove that our new formulas define a cup-$i$ construction, and in \cref{s:outlook} we discuss finer invariants associated to Steenrod squares.