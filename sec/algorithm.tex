%!TEX root = ../fast_sq.tex

\section{New algorithm for Steenrod squares} \label{s:algorithm}

\begin{figure}[b]
	%!TEX root = ../fast_sq.tex

\begin{algorithm} [H]
	\KwIn{$A = \{a_1, \dots, a_m\} \subseteq X_n$}
	$B = \emptyset$\\
	\ForAll{$a_i\ \mathrm{and}\ a_j\ \mathrm{with}\ i < j$}
	{
		$a_{ij} = a_i \cup a_j$\\
		\If{$a_{ij} \in X_{n+k}$}
		{
			$\overline{a}_i = a_i \setminus a_j$\ ;\ \
			$\overline{a}_j = a_j \setminus a_i$\ ;\ \
			$\overline{a}_{ij} = \overline{a}_i \cup \overline{a}_j$ \\
			$\ind \colon \overline{a}_{ij} \to \{0,1\}$ \\
			\ForAll {$v \in \overline{a}_{ij}$}
			{
				$p = \mathrm{position\ of\ } v \mathrm{\ in\ } a_{ij}$\ ;\ \
				$\overline{p} = \mathrm{position\ of\ } v \mathrm{\ in\ } \overline{a}_{ij}$\\
				$\ind(v) = {p}\, +\, \overline{p}\ \ \mathrm{residue\ mod}\ 2$
			}
			\If{\hspace*{3pt}$\ind(\overline{a}_i) \xor \ind(\overline{a}_j)$ = $\{0,1\}$}
			{$B = B \xor \, \{a_{ij}\}$}
		}
	}
	\KwOut{$B \subseteq X_{n+k}$}
	\caption{$SQ^k_n$}
	\label{a:algorithm}
\end{algorithm}
	\caption{Let $X$ be a simplicial complex $X$.
		Passing the support $A \subseteq X_n$ of a cocycle $\alpha$ and an integer $k \in \{1, \dots, n\}$, the algorithm returns the support $B \subseteq X_{n+k}$ of a cocycle representing $Sq^k \big( [\alpha] \big)$.
		We use the notation $S \xor S^\prime = S \cup S^\prime \setminus (S \cap S^\prime)$ and $\ind(S) = \{\ind(v) \mid v \in S\}$.}
	\label{f:algorithm}
\end{figure}

For a finite simplicial complex $X$, integer $k$ and cocycle $\alpha$ of degree $-n$, the cocycle $\beta = (\alpha \ot \alpha)\Delta_{n-k}(-)$ is by
\cref{d:steenrod squares} and \cref{t:main} a representative of $Sq^k \big( [\alpha] \big)$.
In this section we will present and discuss an algorithmic description of $\supp \beta$, the support of $\beta$.

Let $A = \{a_1, \dots, a_m\} \subseteq X_n$ be the support of $\alpha$, which is defined by
\[
\alpha(x) = \begin{cases}
1 & x \in A, \\ 0 & x \not\in A,
\end{cases}
\]
for any $x \in X$.

If $k < 0$ or $k > n$, we have $\beta = 0$ by definition,
so $\supp \beta = \emptyset$.
If $k = 0$, \cref{ex:Sq0 is the identity} shows that $\beta = \alpha$, so $\supp \beta = A$.
For the remaining cases we have the following characterization whose proof occupies \cref{s:correctness}.

\begin{theorem} \label{t:algorithm}
	Let $B$ be the output of \cref{a:algorithm} when the input is $A$ and $k$, then $\supp \beta = B$.
\end{theorem}

We now give an intuitive comparison between our proposed method and a more direct approach using a generic presentation of a cup-$i$ construction
\[
\triangle_i(x) =
\sum_{\Gamma_i} x^{(1)} \ot x^{(2)}.
\]
An algorithm for the computation of the support of $(\alpha \ot \alpha) \triangle_{n-k}(-)$ can be defined by looping over $X_{n+k}$ times $\Gamma_{n-k}$ while evaluating $(\alpha \ot \alpha)$ on the associated tensor pair.
\cref{a:algorithm} improves on this scheme by using the specific form of \eqref{e:new formulas} to filter summands using the support of $\alpha$.
So, even if $X_{n+k}$ and $\Gamma_{n-k}$ are very large, \cref{a:algorithm} loops over
\[
\frac{m(m-1)}{2}
\]
unordered pairs of distinct simplices, where $m$ is the cardinality of $\supp \alpha$.
Many of these pairs are discarded quickly, after checking that the union of its simplices does not have exactly $n+k$ vertices.
%\anibal{It would be great to have an upper bound on the following: Consider $\big\{ a_i \subseteq \N : \bars{a_i} = n\big\}_{i=1}^m$. Bound the cardinality of $\big\{ \{a_i, a_j\} : \bars{a_i \cup a_j} = n+k \big\}$}
One could wonder if the next step in \cref{a:algorithm} -- determining if a resulting set of $n+k$ vertices is a simplex of $X$ -- could slow down the routine significantly.
As illustrated in \cref{s:comparison} through an example, even for a sub-optimal implementation of our algorithm this is not the case.
For high-performance tasks this look-up time could be further reduced by using data structures specialized on the representation of simplicial complexes, but we do not discuss these optimizations here.