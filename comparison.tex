
\section{Performance comparison} \label{s:comparison}

In this section we present a proof-of-concept comparison between the existing method for the computation of Steenrod squares on simplicial complexes, based on Gonz\'alez-D\'iaz--Real's approach \cite[Corollary 3.2]{gonzalez1999combinatorial}, and the one introduced here.
We used a \verb|Python| implementation of our algorithm and the open source computer algebra system \verb|SAGE v9.3.rc3| \cite{sagemath} which includes an implementation of the existing method written by John Palmieri.

Given a topological space $X$ the \textit{suspension} of $X$ is the topological space $\suspension X$ obtained from $X \times [0,1]$ by collapsing $X \times \{0\}$ and $X \times \{1\}$ to points.
The suspension is a very natural construction and for each integer $i \neq 0$ there is an isomorphism $H^i(X) \cong H^{i+1}(\suspension X)$, which can be extended to $i = 0$ by considering reduced cohomology.
A crucial fact about Steenrod squares is that for reduced cohomology with mod 2 coefficients, all operations commuting with the suspension isomorphism are generated by the Steenrod squares.

The real projective plane $\RP^2$, obtained by identifying antipodal points in a sphere, is the simplest space with a non-trivial Steenrod square.
%\footnote{We saw in \eqref{e:Steenrod squares parameterized} that Steenrod squares are parameterized $W(2)_{\S_2}$ which is isomorphic to the cohomology of $\RP^\infty$.}
Its reduced mod 2 cohomology has a single basis element $x_i \in \widetilde{H}^i$ for $i \in \{1, 2\}$ and satisfies $Sq^1(x_1) = x_2$.
We will consider the $i\th$ suspension $\suspension^i \RP^2$ and the non-trivial operation $Sq^1(\suspension^i x_1) = \suspension^i x_2$.

In \verb|SAGE| we use the methods \verb|RealProjectiveSpace(2)| and \verb|suspension(i)| to produce a simplicial complex model of $\suspension^i \RP^2$ for each $i \in \{0, \dots, 10\}$.
Given such model of $\suspension^i \RP^2$ we apply \verb|cohomology_ring(GF(2))| to it and use \verb|basis()| to obtain a model of the element $\suspension^i x_1$.
Finally we apply \verb|Sq(k)| to it with $k=1$ and record the execution time of this last step.
We implemented in \verb|Python| an alternative for the method \verb|Sq(k)| based on \cref{a:algorithm} and modified the above pipeline accordingly.
We recorded the average execution time of these implementations for each $\suspension^i \RP^2$ over $\lfloor 10000/2^i \rfloor$ runs for each $i \in \{0, \dots, 10\}$. We present the results of this analysis in \cref{f:comparison}.
We observe that the execution time of the implementation based on our algorithm is constant whereas the one in \verb|SAGE| grows exponentially.

\begin{figure}
	\includegraphics[width=0.9\textwidth]{comp_sus_rp2.pdf}
	\caption{Average execution time in \texttt{SAGE} of two methods computing Steenrod squares. In orange the one proposed in this article and in blue the one included in \texttt{SAGE v9.3.rc3}. More specifically, for each $i \in \{0, \dots, 10\}$ we timed the computation of the non-trivial Steenrod square in the cohomology of the $i\th$ suspension of the real projective plane, averaged over a number of runs equal to the integral part of $\frac{10000}{2^i}$.}
	\label{f:comparison}
\end{figure}