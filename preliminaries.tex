
\section{Preliminaries} \label{s:preliminaries}

In this section we review the basic notions used in this article and set up the conventions we follow.

\subsection{Chain complexes}

We assume familiarity with the notion of chain complex over a ring $\k$. We will use homological grading regarding any cohomologically graded complex $A$ as homologically graded via $A_n = A^{-n}$.

The \textit{tensor product} $C \otimes C^\prime$ of $C$ and $C^\prime$ is the chain complex whose degree-$n$ part is
\begin{equation*}
\left(C \otimes C^\prime\right)_n = \bigoplus_{i+j=n} C_i \otimes C^\prime_j,
\end{equation*}
where $C_i \otimes C^\prime_j$ is the tensor product of $\k$-modules, and whose boundary map is defined by
\begin{equation*}
\partial (v \otimes w) = \partial v \otimes w + (-1)^{|v|} v \otimes \partial w.
\end{equation*}

The \textit{hom complex} $\Hom(C, C^\prime)$ is the chain complex whose degree-$n$ part is the subset of linear maps between them that increase degree by $n$, i.e.,
\begin{equation*}
\Hom(C, C^\prime)_n = \{f \mid f(C_k) \subseteq C^\prime_{k+n}\},
\end{equation*}
and boundary map defined by
\begin{equation*}
\partial f = \partial_{C^\prime} \circ f - (-1)^{n} f \circ \partial_C.
\end{equation*} 
Notice that a chain map is the same as a $0$-cycle in this complex, and that two chain maps are chain homotopy equivalent if and only if they are homologous cycles. We extend this terminology and say that two maps $f, g \in \Hom(C, C^\prime)$ are \textit{homotopic} if their difference is nullhomologous, referring to a map $h \in \Hom(C, C^\prime)$ such that $\partial h = f - g$ as a \textit{homotopy} between them.

Regarding $\k$ as a chain complex concentrated in degree $0$, the \textit{linear dual} of a chain complex $C$ is the chain complex $\Hom(C, \k)$.
We refer to the contravariant functor $\Hom(-, \k)$ as \textit{linear duality}.

For any three chain complexes, there is a natural isomorphism of chain complexes
\begin{equation} \label{e:adjunction isomorphism}
\Hom(C \otimes C^\prime, C^{\prime\prime}) \cong
\Hom(C, \Hom(C^\prime, C^{\prime\prime}))
\end{equation}
referred to as the \textit{adjunction isomorphism}.

\subsection{Group actions}

Symmetries on chain complexes play an important role on this work.
Let $G$ be a finite group.
We will later focus solely on the symmetric group $\S_2$.
We denote by $\k[G]$ the group ring of $G$, i.e., the free $\k$-module generated by $G$ together with the ring product defined by linearly extending the product of $G$.
We refer to a chain complex of left $\k[G]$-modules as a chain complex with a $G$-\textit{action} and to $\k[G]$-linear maps as $G$-\textit{equivariant}.

Given a chain complex $C$ with a $G$-action we naturally associate the following two chain complexes.
The subcomplex of \textit{invariant chains} of $C$, denoted $C^G$, contains all elements $c \in C$ satisfying $g \cdot c = c$ for every $g \in G$.
The quotient complex of \textit{coinvariant chains} of $C$, denoted $C_G$, is the chain complex obtained by identifying elements $c, c^\prime \in C$ if there exists $g \in G$ such that $c^\prime = g \cdot c$.

Let $C$ and $C^\prime$ be chain complexes and assume $C$ has a $G$-action.
The chain complex $\Hom(C, C^\prime)$ has a $G$-action induced from $(g \cdot f)(c) = f^{-1}(g^{-1} \cdot c)$ and there is an isomorphism
\begin{equation} \label{e:invariant hom iso hom coinvariants}
\Hom(C, C^\prime)^G \cong \Hom(C_G, C^\prime).
\end{equation}

\subsection{Simplicial topology}

Simplicial complexes are used to combinatorially encode the topology of spaces.
An \textit{abstract and ordered simplicial complex}, or a \textit{simplicial complex} for short, is a pair $(V, X)$ with $V$ a poset and $X$ a set of subsets of $V$ such that: 
\begin{enumerate}
	\item The restriction of the partial order of $V$ to any element in $X$ defines a total order on it.
	\item For every $v$ in $V$, the singleton $\{v\}$ is in $X$.
	\item If $x$ is in $X$ and $y$ is a subset of $x$, then $y$ is in $X$.
\end{enumerate}
We abuse notation and denote the pair $(V, X)$ simply by $X$ referring to $V$ as its poset of vertices.

The elements of $X$ are called \textit{simplices} and the \textit{dimension} of a simplex $x$ is defined by subtracting $1$ from the number of vertices it contains.
Simplices of dimension $n$ are called $n$-simplices and are denoted by their order set of vertices $[v_0, \dots, v_n]$.
The collection of $n$-simplices of $X$ is denoted $X_n$.
There are natural maps $d_i^n \colon X_n \to X_{n-1}$ for $i \in \{0, \dots, n\}$ defined by
\begin{equation*}
d_i^n([v_0, \dots, v_n]) = [v_0, \dots, \widehat{v}_i, \dots, v_n]
\end{equation*}
and referred to as the $i^{\th}$ face map in dimension $n$.
These satisfy the \textit{simplicial relation}:
\begin{equation} \label{e:simplicial relation}
d_i^{n-1} d^n_j = d_{j-1}^{n-1} d_i^n
\end{equation}
for any $0 \leq i < j \leq n$.
We will omit the superscripts of these maps when no confusion arises from doing so.

A \textit{simplicial map} $X \to X^\prime$ is a poset morphisms between their vertices $f \colon V \to V^\prime$ sending simplices to simplices, i.e., satisfying that if $[v_0, \dots, v_n] \in X$ then $[f(v_0), \dots, f(v_n)] \in X^\prime$.

Let $X$ be simplicial complex.
The degree-$n$ part of the chain complex of \textit{chains} of $X$, which we denote $\chains(X; \k)$, is defined by
\begin{equation*}
C_n(X; \k) = \k \big\{ X_n \big\},
\end{equation*}
i.e., the $\k$-module freely generated by the $n$-dimensional simplices of $X$.
The boundary map is defined on basis elements by
\begin{equation*}
\begin{tikzcd}[column sep=normal, row sep=tiny,row sep = 0ex
,/tikz/column 1/.append style={anchor=base east}
,/tikz/column 2/.append style={anchor=base west}
]
C_n(X; \k) \arrow[r, "\partial_n"] & C_{n-1}(X; \k) \\
x\ \arrow[r, |->] & \sum_{i=0}^{n} (-1)^i d_i (x).
\end{tikzcd}
\end{equation*}

Given a simplicial map $f \colon X \to X^\prime$, the \textit{induced map} $f_\bullet \colon \chains(X; \k) \to \chains(X^\prime; \k)$ is the chain map defined on basis elements by $f_\bullet([v_0, \dots, v_n]) = [f(v_0), \dots, f(v_n)]$ if $i \neq j$ implies $ f(v_i) \neq f(v_j)$ and it is $0$ otherwise.

We referred to
\begin{equation*}
\cochains(X; \k) = \Hom(\chains(X; \k), \k)
\end{equation*}
as the \textit{cochains} of $X$ and to the dual $\delta^{n}$ of $\partial_{-n}$ as the $n\th$ coboundary map.
Furthermore, we denote the linear dual of the map $f_\bullet$ induced by a simplicial map $f$ by $f^\bullet$.
We remark that $C_n(X; \k) = 0$ for $n < 0$ and $C^n(X;\k) = 0$ for $n > 0$.

The \textit{homology} $H_\bullet(X; \k)$ and \textit{cohomology} $H^\bullet(X; \k)$ of $X$ are defined by applying the kernel-mod-image construction to $\chains(X; \k)$ and $\cochains(X; \k)$ respectively.
More specifically, a (co)homology class in degree $n$ is an equivalence class of elements in the kernel of the $n\th$ (co)boundary map, referred to as  $n\th$\textit{(co)cycles}, where two are regarded as equivalent if their difference is in the image of the $(n+1)\th$ (co)boundary map.

When $X$ and $\k$ are clear from the context we omit them from the notation.