
\section{New algorithm} \label{s:algorithm}

Given a finite simplicial complex $X$ we now present an algorithm for the determination of a cocycle representative of $Sq^k([\alpha])$ given a representative cocycle $\alpha$ in degree $-n$.

If $k$ is a negative integer or greater than $n$ we $Sq^k([\alpha]) = 0$, and if $k = 0$ we we have $Sq^k([\alpha]) = [\alpha]$ which are uninteresting cases, so we assume $k$ is between $1$ and $n$.

Consider $A = \{a_1, \dots, a_m\} \subseteq X_n$ with $\alpha = \alpha_1 + \cdots + \alpha_m$ such that
\begin{equation*}
\alpha_i(a_j) = \begin{cases}
1 & i=j, \\ 0 & i\neq j,
\end{cases}
\end{equation*}
and let $B \subseteq X_{n+k}$ be the output of algorithm $SQ_n^k$ applied to $A$, please consult Figure~\ref{f:algorithm}.
\begin{figure}[h]
	\begin{algorithm} [H]
	\KwIn{$A = \{a_1, \dots, a_m\} \subseteq X_d$}
	$B = \emptyset$\\
	\ForAll{$a_i\ \mathrm{and}\ a_j\ \mathrm{with}\ i < j$}
	{	
		$a_{ij} = a_i \cup a_j$\\
		\If{$a_{ij} \in X_{d+k}$}
		{
			$\overline{a}_i = a_i \setminus a_j$\ ;\ \ 
			$\overline{a}_j = a_j \setminus a_i$\ ;\ \
			$\overline{a}_{ij} = \overline{a}_i \cup \overline{a}_j$ \\
			$index =$ empty dictionary\\
			\ForAll {$v \in \overline{a}_{ij}$}
			{
				$p = \mathrm{position\ of\ } v \mathrm{\ in\ } a_{ij}$\ ;\ \
				$\overline{p} = \mathrm{position\ of\ } v \mathrm{\ in\ } \overline{a}_{ij}$\\
				$index[v] = {p}\, +\, \overline{p}\ \ \mathrm{residue\ mod}\ 2$
			}
			\If{\hspace*{3pt}\{$index[v] \mid \mathrm{in}$ $\overline{a}_i$\} $\mathrm{xor}$ \{$index[v] \mid v$ $\mathrm{in}$ $\overline{a}_j$\} = $\{0,1\}$}
				{$B = B$ xor $\{a_{ij}\}$}
		}		
	}
	\KwOut{$B$}
	\caption{$SQ^k$}
\end{algorithm}
	\caption{Algorithm producing for a simplicial complex $X$, non-negative integer $n$, integer $k$ between $1$ and $n$, and cocycle $\alpha$ presented as a set $A \subseteq X_n$, a cocycle representing $Sq^k([\alpha])$ returned as a set $B \subseteq X_{n+k}$.
	We use the notation $S \xor S^\prime = S \cup S^\prime \setminus (S \cap S^\prime)$.}
	\label{f:algorithm}
\end{figure}
We will prove the correctness of this algorithm by showing that if $\beta = (\alpha \otimes \alpha)\Delta_{n-k}(-)$, then
\begin{equation*}
\beta(x) =
\begin{cases}
1 & x \in B, \\
0 & x \not\in B.
\end{cases}
\end{equation*}

\begin{lemma} \label{l:freeness}
	For any $x \in X_{n+k}$ and $i,j \in \{1,\dots,m\}$ we have:
	\begin{enumerate}
		\item if $i=j$, then $(\alpha_i \otimes \alpha_j)\Delta_{n-k}(x) = 0$,
		\item if $i \neq j$ and $(\alpha_i \otimes \alpha_j)\Delta_{n-k}(x) \neq 0$, then $(\alpha_j \otimes \alpha_i)\Delta_{n-k}(x) = 0$.
	\end{enumerate}
\end{lemma}

\begin{proof}
	Let $U, V \subseteq \{0, \dots, n\}$ be any pair of subsets. If at least one of them is not empty, then $U^+ \neq V^-$.
\end{proof}

For any $x \in X_{n+k}$ the value of $\beta$ evaluated on the chain represented by $x$ is given by 
\begin{align*}
\beta(x) & =
(\alpha \otimes \alpha) \Delta_{n-k}(x) \\ & =
(\alpha_1 + \cdots + \alpha_m)^{\otimes 2} \Delta_{n-k}(x) \\ & =
\Big(\sum_{i \neq j} \alpha_i \otimes \alpha_j + \sum_{i} \alpha_i \otimes \alpha_i \Big)
\Delta_{n-k}(x) \\ & =
\Big(\sum_{i \neq j} \alpha_i \otimes \alpha_j\Big)
\Delta_{n-k}(x).
\end{align*}
By Lemma~\ref{l:freeness}, at most one of the terms $\alpha_i \otimes \alpha_j$ and $\alpha_j \otimes \alpha_i$ evaluates to $1$ on $\Delta_{n-k}(x)$, so to determine the value of $\beta(x)$ we need to count mod 2 the number of two element subsets $\{i,j\} \subseteq \{1,\dots, m\}$ such that $\{a_i, a_j\} = \{d_{U^+}(x), d_{U^-}(x)\}$ for some $U \subseteq \{0, \dots, n\}$ of cardinality $k$.
This condition on the pair $\{a_i, a_j\}$ implies in particular that $a_i \cup a_j = x$ since $d_{U^+}(x) \cup d_{U^-}(x) = x$.

Let us formalize these observations.
Let $\widetilde{A}$ be the set of all two element subsets $\{a_i, a_j\} \subseteq A$ with $a_i \cup a_j \in X_{n+k}$.
There is a natural map $\pi \colon \widetilde{A} \to X_{n+k}$ sending $\{a_i, a_j\}$ to $a_i \cup a_j$.
Let $\eval \colon A \to \F_2$ be the map sending $\{a_i, a_j\}$ to $1$ if there exist $U \subseteq \{0,\dots,n\}$ of cardinality $k$ such that $\{a_i, a_j\} = \{d_{U^+}(x), d_{U^-}(x)\}$ and let $\eval(\{a_i, a_j\}) = 0$ otherwise.
Notice that if such $U$ exists then it is unique.
With this notation we have
\begin{equation*}
\beta(x) = \sum_{\pi^{-1}(x)} \eval(\{a_i, a_j\}).
\end{equation*}
We will give a computational characterization of $\eval(\{a_i, a_j\}) = 1$ after introducing some more notation.

Given a finite totally ordered set $S$, the \textit{position function} $\pos_S \colon S \to \Z$ sends an element $s \in S$ to the cardinality of $\{s^\prime \in S \mid s^\prime \leq s\}$.
Denote
\begin{equation*}
\overline{a}_{i} = a_i \setminus a_j, \qquad
\overline{a}_{i} = a_i \setminus a_j, \qquad
\overline{a}_{ij} = \overline{a}_i \cup \overline{a}_j.
\end{equation*}
Given $\{a_i, a_j\} \in \pi^{-1}(x)$, there exists $U$ with $\{a_i, a_j\} = \{d_{U^+}(x),\, d_{U^-}(x)\}$
if and only if $U = \pos_{x}(\overline{a}_{ij})$ and
\begin{equation} \label{e:pos's equal U's}
\big\{\pos_{x}(\overline{a}_i),\, \pos_{x}(\overline{a}_j)\big\} = \{U^+, U^-\}.
\end{equation}
Recall that $U^+$ (resp. $U^-$) is the subset of $U \subseteq \{0, \dots, n+k\}$ with elements satisfying that their position in $U$ plus its position in $\{0, \dots, n+k\}$ is $0$ mod 2 (resp. $1$ mod 2).
Therefore, \eqref{e:pos's equal U's} is equivalent to the function $\ind \colon \pos_x(\overline{a}_{ij}) \to \F_2$ defined by
\begin{equation*}
\ind(v) = \pos_{x}(v) + \pos_{\overline{a}_{ij}}(v) \text{ mod }2
\end{equation*}
being constant on both $\overline{a}_i$ and $\overline{a}_j$ with different values, i.e., 
\begin{equation*}
\ind(\overline{a}_i) \xor \ind(\overline{a}_j) = \{0,1\}.
\end{equation*}

Therefore, for any $x \in X_{n+k}$ we have $\beta(x) = 1$ if and only if the set of pairs $\{a_i, a_j\}$ in $\pi^{-1}(x)$ with $\eval(\{a_i, a_j\}) = 1$ is even; which is equivalent to the condition $x \in B$.